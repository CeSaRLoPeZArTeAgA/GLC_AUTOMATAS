%\documentclass[12pt]{article}
\documentclass[12pt, twoside, a4paper]{article}
\usepackage[top=2.54cm, bottom=2.54cm, left=2.54cm, right=2.54cm]{geometry}
% === Fuente Arial y formato ===
%\usepackage[scaled=0.92]{helvet} % Helvetica ≈ Arial
%\renewcommand{\familydefault}{\sfdefault} % Usa fuente sans-serif (Arial)
\usepackage[T1]{fontenc}
\usepackage[utf8]{inputenc}
\usepackage[spanish]{babel}
\usepackage{setspace}
\setstretch{1} % Espaciado simple

\usepackage{amsthm} % Importante para los entornos de teoremas y pruebas
%\usepackage{amsmath}
\usepackage{hyperref}
%\usepackage{amssymb}
\usepackage{amsmath, amsfonts, amssymb}
\usepackage{scalefnt}
\usepackage{enumitem}
\setlist{nosep}
\usepackage{xcolor}
\usepackage{mathtools}
\usepackage{float}
\usepackage{titlesec} % Opcional para ajustar el estilo de las secciones
\usepackage{graphicx}
\usepackage{layout}
\usepackage{titlesec}
\usepackage{tabularx}
\usepackage{booktabs}
\usepackage{minted} % Cargar el paquete minted
\usepackage[table,xcdraw]{xcolor}
\usepackage{blindtext}
\usepackage{geometry}
\usepackage{placeins}
\usepackage{subcaption}
\usepackage{circuitikz}
\usepackage{steinmetz}
\usepackage[utf8]{inputenc}   % casi siempre ya está
\usepackage[T1]{fontenc}
\usepackage{listings}
\lstset{literate={ε}{{$\varepsilon$}}1 {δ}{{$\delta$}}1  {ξ}{{$\xi$}}1 {≤}{{$\leq$}}1  {ρ}{{$\rho$}}1 {∞}{{$\infty$}}1 {γ}{{$\gamma$}}1 }

\usepackage{xcolor} % Para colores si se desean resaltar elementos

\usepackage{circledsteps}%encerrar un numero con circulo
\usepackage{ulem}%raya abajo del numero
\usepackage[spanish]{babel} % Para el idioma español
\usepackage{xcolor} % Para colores

\lstset{
    basicstyle=\ttfamily\footnotesize,
    keywordstyle=\color{blue},
    stringstyle=\color{green!50!black},
    commentstyle=\color{gray},
    numberstyle=\color{cyan},
    identifierstyle=\color{black},
    backgroundcolor=\color{gray!10},
    showstringspaces=false,
    breaklines=true,
    frame=single,
    numbers=left,
    tabsize=4,
    captionpos=b,
    caption={Algoritmo 1 Método de Penalización/Barrera Modificada},
    label=alg:penalty_barrier_espanol
}


\usepackage{algorithm}
\usepackage{algorithmic}



\usepackage{amsmath}
\newtheorem{theorem}{Teorema}[section]
\newtheorem{lemma}[theorem]{Lema}
\newtheorem{proposition}[theorem]{Proposición}
\newtheorem{corollary}[theorem]{Corolario}
\newtheorem{definition}[theorem]{Definición}
% Personalización del entorno de pruebas con el cuadro en una nueva línea
\renewenvironment{proof}[1][\textbf{Prueba:}]{\par\noindent#1 \par\vspace{0.5em}}{\par\vspace{1em}\hfill$\Box$\par}



\geometry{a4paper,left = 24mm,top = 20mm}
\usepackage{enumerate}
\usepackage{setspace}
\usepackage{caption}
\setstretch{1.5} %
\usepackage{tikz}
\usetikzlibrary{er,positioning}
\setstretch{1.5} 
\begin{document}
\begin{titlepage}
	%\layout
    \voffset0cm
	  \begin{center}
		\vspace*{1 mm}
        {\LARGE \textbf{UNIVERSIDAD NACIONAL DE}}\\
            \vspace{3mm} % Aumentamos el espacio vertical
        {\LARGE \centering\textbf{INGENIERÍA}}\\
        \vspace{6mm}
		{\LARGE\textsc {Facultad de Ciencias}}\\
		\vspace{2 mm}

        {\textsc {Escuela Profesional de Ciencias de la Computación}}\\
		\vspace{4 mm}
        \begin{center}
            \includegraphics[scale = 0.16]{LogoUni.png}
            
        \end{center}
        \LARGE\textsc {}
        \\
        {\LARGE\textsc {Teoría de Autómatas, Lenguajes y Computación - CC321A}}\\
        {\large \textbf{ Practica Nº4 : }\\
        {\large \textbf{.....}}
      

		\vspace{1 mm}
		{\large \textbf{Integrantes:} }\\
	    {\fontsize{12}{1} \selectfont - Berdiales Diaz, Maurizio - 20202113E }\\
        {\fontsize{12}{1} \selectfont - Muñoz Diaz, Jose Estalin - 20222718J }\\
        {\fontsize{12}{1} \selectfont - López Arteaga, César Omar - 20215508C }\\
        {\fontsize{12}{1} \selectfont - Arévalo Gambini, Alois Diógenes - 20172759J }\\
         

\vspace{3mm}
        {\large \textbf{Profesor: Victor Andres Melchor Espinoza}}\\
        \vspace{1 mm}
	\end{center}
\end{titlepage}

\newpage
    \renewcommand{\contentsname}{Contenido}
    \tableofcontents
\newpage

%%%%%%%%%%%%%%%%%%%%%%%%%%%%%%%%%%%%%%%%%%%%%%%%%%%%%%%%%%%%%%%%%%%%%%%%%%



\section*{Gramáticas}

\noindent Las gramáticas utilizadas para este proyecto se denominan gramáticas libres de contexto, en las que cada regla tiene un solo no terminal en el lado izquierdo de la flecha $\rightarrow$.\\

\noindent Considere las siguientes producciones para la gramática $G$:
\textbf{\textit{}}
\begin{align*}
\langle \text{inicio} \rangle 
    &\rightarrow \langle \text{historia} \rangle . \\[4pt]
\langle \text{historia} \rangle 
    &\rightarrow \langle \text{frase} \rangle 
       \mid \langle \text{frase} \rangle \; \text{y} \; \langle \text{historia} \rangle 
       \mid \langle \text{frase} \rangle \; \text{sino} \; \langle \text{historia} \rangle \\[4pt]
\langle \text{frase} \rangle 
    &\rightarrow \langle \text{articulo} \rangle \;
       \langle \text{sustantivo} \rangle \;
       \langle \text{verbo} \rangle \;
       \langle \text{articulo} \rangle \;
       \langle \text{sustantivo} \rangle \\[4pt]
\langle \text{articulo} \rangle 
    &\rightarrow \text{el} \mid \text{la} \mid \text{al} \\[4pt]
\langle \text{sustantivo} \rangle 
    &\rightarrow \text{gato} \mid \text{niño} \mid \text{perro} \mid \text{niña} \\[4pt]
\langle \text{verbo} \rangle 
    &\rightarrow \text{perseguía} \mid \text{besaba}
\end{align*}

\noindent Así, puede generar la siguiente historia empezando del símbolo no terminal $\langle inicio \rangle$:

\begin{quote}
el gato perseguía al niño y el niño besaba al gato.
\end{quote}


%%%%%%%%%%%%%%%%%%%%%%%%%%%%%%%%%%%%%%%%%%%%%%%%

\section*{Enteros PseudoAleatorios}

\noindent El algoritmo de Park--Miller (llamado así por sus inventores) es una forma sencilla de generar una secuencia de términos enteros pseudoaleatorios. Funciona así. Sea $N_0$ un número entero llamado semilla. La semilla es el primer término de la sucesión y debe estar entre $1$ y $2^{31}-2$, inclusive. A partir de la semilla, los términos posteriores $N_1, N_2, \dots$ se producen mediante la siguiente ecuación:

\[
N_{k+1} = (7^5 N_k) \bmod (2^{31}-1)
\]

\noindent Aquí $7^5 = 16807$ y $2^{31} = 2147483648$. El operador \% de Python devuelve el resto después de dividir un número entero por otro.\\


\noindent Siempre obtendrá la misma secuencia de términos de una semilla determinada. Por ejemplo, si comienza con la semilla $101$, obtendrá una secuencia pseudoaleatoria cuyos primeros términos son:

\[
1697507,\; 612712738,\; 678900201,\; 695061696,\; 1738368639,\; 
246698238,\; 1613847356,\; 1214050682.
\]

\noindent Podría usar esta secuencia para probar si su generador de números aleatorios funciona correctamente.\\

\noindent Los términos de la secuencia pueden ser grandes, pero puede hacerlos más pequeños usando de nuevo el operador \%. Si $N$ es un término de la sucesión y $M$ es un entero mayor que $0$, entonces $N \bmod M$ da un número entero entre $0$ y $M-1$, inclusive. Por ejemplo, si necesita un número entre $0$ y $9$, entonces escribiría $N \bmod 10$.\\

\noindent Escriba una aplicación en Python que genere cadenas aleatorias usando una gramática. Debe implementar las siguientes clases:

%%%%%%%%%%%%%%%%%%%%%%%%%%%%%%%%%%%%%%%%%%%%%%%%%

\section*{a.- La clase Aleatorio}

\noindent La primera clase debe denominarse \texttt{Aleatorio} y debe implementar el algoritmo de Park--Miller. Debe tener los siguientes métodos:

\subsection*{\_\_init\_\_(self, semilla)}

\noindent Inicializa una instancia de la clase para que genere la secuencia de enteros pseudoaleatorios que comienzan con \textit{semilla}. La semilla es un entero en el rango adecuado para el algoritmo.

\subsection*{siguiente(self)}
\noindent Genera el siguiente entero aleatorio en la secuencia y lo devuelve.

\subsection*{elegir(self, limite)}
\noindent Llama al método \texttt{siguiente} para obtener un número entero aleatorio.  
Luego calcula un nuevo entero entre $0$ y \texttt{limite}$-1$. Devuelve este nuevo entero.


%%%%%%%%%%%%%%%%%%%%%%%%%%%%%%%%%%%%%%%%%%%%%%%%%%

\section*{b.- La clase Regla}

\noindent Representa una regla de una gramática. Debe tener el siguiente constructor:

\subsection*{\_\_init\_\_(self, izquierda, derecha)}
\noindent La variable \texttt{self.left} debe ser la cadena del lado izquierdo.  
La variable \texttt{self.right} debe ser la tupla del lado derecho.  
La variable \texttt{self.cont} debe inicializarse en el entero $1$.

\subsection*{\_\_repr\_\_(self)}
\noindent Devuelve una cadena de la forma  
\[
"C\; L \rightarrow R_1\; R_2 \dots R_n"
\]
donde $C$ es \texttt{self.cont}, $L$ es \texttt{self.left} y $R_i$ son los elementos de \texttt{self.right}.

%%%%%%%%%%%%%%%%%%%%%%%%%%%%%%%%%%%%%%%%%%%%%%%
\section*{c.- La clase Gramatica}

\noindent Esta clase implementa una gramática usando reglas como las anteriores.

\subsection*{\_\_init\_\_(self, semilla)}
\noindent Crea una instancia de \texttt{Aleatorio} usando \textit{semilla}.  
Define un diccionario vacío para almacenar reglas.

\subsection*{regla(self, izquierda, derecha)}
\noindent Agrega una nueva regla a la gramática.  
\textit{izquierda} es la cadena del lado izquierdo.  
\textit{derecha} es una tupla de cadenas (terminales y no terminales).

\noindent Si la clave no existe, se crea como una tupla con una única regla.  
Si existe, se agrega al final.

\subsection*{generar(self)}
\noindent Genera una cadena comenzando desde la regla cuyo lado izquierdo es \texttt{'Inicio'}.  
Si no existe, genera una excepción.

\subsection*{generando(self, strings)}
\noindent El parámetro \texttt{strings} es una tupla de cadenas terminales y no terminales.  
Se recorre cada cadena; si es terminal se agrega al resultado; si es no terminal se selecciona una regla y se genera recursivamente.

\subsection*{seleccionar(self, left)}
\noindent Elige aleatoriamente una regla cuyo lado izquierdo sea \textit{left} usando variables de conteo y el generador aleatorio.




%%%%%%%%%%%%%%%%%%%%%%%%%%%%%%%%%%%%%%%%%%%%%%%%%%%%%%%%%%%%%%%%%%%%%%%%%%%%%%%%%%%%%%

\end{document}